%%%%%%%%%%%%%%%%%%%%%%%%%%%%%%%%%%%%%%%%%%
% Master Thesis 
% Polina Polunina
% November 2022 
%
% License:
% CC-BY-SA 4.0 -- Creative Commons Attribution-ShareAlike 4.0 International
% https://creativecommons.org/licenses/by-sa/4.0/legalcode
%%%%%%%%%%%%%%%%%%%%%%%%%%%%%%%%%%%%%%%%%%
\section{Discussion and Outlook}

In this thesis, two methods for SARS-CoV-2 wastewater surveillance have been proposed as two workflows. One workflow is assumed to be used for standalone data analysis of wastewater samples extracted using a metatranscriptomics-based library preparation technique and sequenced using Illumina sequencing approach. This workflow has one branch with Freyja tool on the step of SARS-CoV-2 lineages abundances computation. At the same time, the second workflow is developed for standalone data analysis of wastewater samples obtained with an ampliconic-based technique and Illumina sequencing approach. Compare to the first workflow suggested, this workflow has two branches with Freyja and COJAC tools accordingly, on the step of SARS-CoV-2 lineages abundances computation.

Both workflows were examined on synthetic and real-world datasets to evaluate efficiency in finding SARS-CoV-2 lineages abundances and their proportions in samples. The results obtained with developed workflows were additionally benchmarked with one state-of-the-art workflow called Lineagespot. The results acquired with developed Galaxy workflows using Freyja and COJAC approaches are promising.

Developed Galaxy workflows with Freyja-based lineages abundances computation showed promising results in detecting expected lineages in cases of single lineage or two lineages were expected. However, the results were biased by detecting other unexpected lineages. There were detected unexpected lineages in all samples of the mock dataset, while they should not have appeared. The other Freyja-based method limitations are due to the fact that for two out of four examined real-world datasets (more precisely, for Californian and UK ones) all detected SARS-CoV-2 variants were labeled as “Other” when they should have been labeled by WHO designation names. This led to non-informative plots which rely on these labels and the value of SARS-CoV-2 variant proportions corresponding to these WHO designation labels. Nonetheless, lineages were detected for all real-world datasets using Pango naming system.

Galaxy workflow with COJAC implemented for delineation showed slightly more efficient results, compared to Freyja-based workflows. It was able to detect only lineages that were expected while experimenting on mock dataset. On mock samples where only a single lineage was expected, the proposed COJAC-based method showed more effective results, compared to the proposed Freyja-based method, but less effective, compared to Lineagespot approach. Worth it to say, that on mock samples, where only two lineages were expected, COJAC-based method showed better results, compared to Lineagespot, in detecting these two lineages, however, results were biased with detecting unexpected lineages (as it was for Freyja-based). As for other disadvantages of COJAC-based method, it works on amplicons only; thus, this method cannot be used for metatranscriptomes. Moreover, COJAC tool does not provide integrated visualization of its results like Freyja; hence, Jupyter Notebook was used outside of developed workflows to evaluate results on real-world datasets.

Due to both Freyja-based and COJAC-based workflows' shortcomings, listed above, there is room for improvement. Throughout the thesis, a few starting points for further improvement of both methods were identified.

One starting point for improvement is, because of Freyja limits on WHO variant name designation for further plotting of results, visualization can be implemented in Galaxy workflow independently. One option for that can be an addition of extra steps in developed Galaxy workflows using Jupyter notebooks in Galaxy \cite{galaxyjupyter,batut2018}. Galaxy opens the opportunity of using Jupyter notebooks in Galaxy workflows. 

The other idea for improvement of developed Galaxy workflows focused on COJAC-based branch. The visualization is currently missing, thus, an extra visualization step is needed to be implemented. However, in the case of COJAC the suggestion for further improvement differs from the one for Freyja. COJAC outputs can be visualized in CoV-Spectrum \cite{chen2022b} format and uploaded to this platform. This can be accomplished with the help of Python scripts as it was done for Swiss dataset analysis by Jahn et al. \cite{jahn2022,swisscovspectrum}. Unfortunately, code is not provided under open source license, thus, currently cannot be used for the improvement of Galaxy workflows. 

The potential next step for enhancement could be to create tutorials based on Galaxy training materials \cite{batut2018} that will cover how to use the developed Galaxy workflows for SARS-CoV-2 wastewater surveillance step by step and an explanation of what each step accomplishes. The further step would be to connect them to the automatic Galaxy bot to analyze data on a regular basis in a similar way it is currently managed for clinical data.

Wastewater surveillance has proved to be efficient in the early detection of viruses and its lineages which is crucial for the early detection of newly emerged variants. Even though this thesis focused on SARS-CoV-2 wastewater samples analysis, it is not a limit. As an option for further improvement of proposed pipelines, surveillance can be extended to other viruses. For example, wastewater surveillance is already practiced extensively for detecting poliovirus in the US, more specifically, in New York \cite{russo2022}, and plays an essential role in global health. It is desirable to expand wastewater surveillance methods to other pathogens beyond SARS-CoV-2 and validate them with cases-based epidemiological data. To be applicable to locations without centralized sewer infrastructure, current methods must be adapted and optimized [88]. To fully exploit the potential of wastewater surveillance for global pathogen surveillance, international sharing of wastewater-based pathogen sequencing data will be required.

\clearpage
