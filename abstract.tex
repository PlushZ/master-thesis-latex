%%%%%%%%%%%%%%%%%%%%%%%%%%%%%%%%%%%%%%%%%%
% Master Thesis 
% Polina Polunina
% October 2022 
%
% License:
% CC-BY-SA 4.0 -- Creative Commons Attribution-ShareAlike 4.0 International
% https://creativecommons.org/licenses/by-sa/4.0/legalcode
%%%%%%%%%%%%%%%%%%%%%%%%%%%%%%%%%%%%%%%%%%
\section*{Abstract}

More than 630 million people have been affected by the COVID-19 pandemic nearly two years after the first report of SARS-CoV-2 in Wuhan, China. During the SARS-CoV-2 pandemic, wastewater surveillance has received extensive public attention as a passive monitoring system that complements clinical and genomic surveillance. The detection and quantification of viral RNA in wastewater samples are already possible through several methods and protocols, and viral RNA concentrations in wastewater have been shown to correlate with reported cases.
    
The Galaxy community has put much effort into a continuous analysis of intra-host variation in SARS-CoV-2, including the development of workflows, on samples of individuals.
    
In this master thesis, there were investigated existing Galaxy workflows for clinical SARS-CoV-2 data analysis, as well as existing approaches to wastewater surveillance of SARS-CoV-2. Two existing Galaxy workflows and, on the other hand, two existing wastewater surveillance methods (Freyja and COJAC) were used to develop Galaxy pipelines for SARS-CoV-2 wastewater data analysis. The resulting workflows were tested on synthetic and real-world datasets to evaluate efficiency in finding SARS-CoV-2 lineages abundances. The developed workflow was additionally compared with one state-of-the-art approach called Lineagespot.

Results of the Galaxy workflows developed using the Freyja and COJAC approaches are encouraging. Developed Galaxy workflows with Freyja showed promising results in detecting expected lineages when single lineage or two lineages were expected. However, the results were biased by detecting other unexpected lineages. Galaxy workflow with COJAC implemented for delineation showed slightly more efficient results as it was able to detect only lineages that were expected in some samples.


